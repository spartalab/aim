\documentclass[11pt]{article}
\begin{document}

We now need to determine the time of this current all-red-signal phase. Assume we have known the $TL$ as trafficLevel and $HP$ as humanPercentage by observing the traffic. $laneNum$ is defined as the number of entering lanes. We define active lanes as the lanes which the leading vehicle is not human-driven. The expectation of number of active lanes, $EAL$, can be calculated as
\begin{equation}
EAL=\Sigma_{i=0}^{laneNum} i\cdot C_{laneNum}^i\cdot (1-HP)^i \cdot HP^{laneNum-i}
\end{equation}

What we need to consider is how many vehicles are able to enter the intersection during the all-red-signal phases. We define these vehicles as active vehicles. Intuitively, they are autonomous vehicles, if any, that in front of the first human-driven vehicles in every lanes. They can be defined strickly as follows,
\begin{itemize}
\item Active vehicles are autonomous ones in active lanes.
\item No human-driven vehicles are in front of any active vehicle.
\item No human-driven vehicles appear between any two active vehicle in the same lane.
\end{itemize}
Therefor, the expection of number of active vehicles for each lane, $EAV$, can be cacluated approximately as
% This one need to more simplified
\begin{equation}
EAV=\Sigma_{i=2}^\infty i \cdot HP^{i-1}
\end{equation}

Then, we want to get the time for these active vehicles to go throught the intersection. This can be determined by the result of effective traffic level versus delay time. Consider the the signals of the lanes before they are all red. If one lane is previously green, we assume the traffic level of this lane is exactly the traffic level observed, $TL$. If one lane is previouls red, we need to determine whether there's a conjestion here. This can be done by checking the result of traffic level versus throughput. If the traffic level at this point equals its throughput, we assume the traffic level of this lane is the observed one, $TL$. Otherwise, we assume it's max traffic level. So, we can define delay time for one lane, $delayOneLane$, as follows
\begin{equation}
delayOneLane=\left\{
\begin{array}{lcl}
EAV \cdot delay(TL) && {,if\ green\ lane}\\
EAV \cdot delay(maxTrafficLevel) && {,if\ red\ lane}
\end{array} \right.
\end{equation}

To let all the active vehicles clear up, we need to find the maximum delay time. We assume that the longest delay time determine how long the ative vehicles in this intersection would be cleared, even thought they are affected by other lanes.

\end{document}